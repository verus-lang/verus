\documentclass[11pt,twocolumn]{article}
\usepackage[margin=1.5cm,columnsep=1cm]{geometry}
\usepackage{titling}
\usepackage[tiny]{titlesec}
\usepackage[utf8]{inputenc}
\usepackage{times}
\usepackage{listings, listings-rust}
\usepackage{caption}
\usepackage{subcaption}
\usepackage{float}
\usepackage{tikz}

\newcommand{\codesize}{\fontsize{8pt}{9pt}\selectfont}
\newcommand{\textcode}[1]{{\codesize\ttfamily{#1}}}
\newcommand{\divider}[0]{\begin{center}\rule{0.48\linewidth}{0.48pt}\end{center}}
\titlespacing{\paragraph}{0pt}{0.5\baselineskip}{1em}
\renewcommand{\section}[1]{\paragraph{#1.}}

\lstset{
basicstyle=\codesize\ttfamily,
escapeinside={\%*}{*}
}

\newcommand*\circled[1]{\,\tikz[baseline=(char.base)]{
            \node[shape=circle,draw,inner sep=.7pt] (char) {\textcode{#1}};}\,}
\newcounter{clnumcounter}
% \counterwithin{clnumcounter}{figure}

\renewcommand{\theclnumcounter}{\Alph{clnumcounter}}
\newcommand{\clnum}{\refstepcounter{clnumcounter}\circled{\theclnumcounter}}
\newcommand{\clref}[1]{\circled{\ref{#1}}}

\predate{}
\postdate{}
\pretitle{\begin{flushleft}\Large\vspace{-3em}}
\posttitle{\end{flushleft}}
\preauthor{\begin{flushleft}}
\postauthor{\end{flushleft}\vspace{-1.2em}\rule{2cm}{0.2pt}\vspace{-.8em}}
\date{}

\title{Verus --- SMT-based verification of low-level systems code\\\small github.com/verus-lang/verus}

\author{Andrea Lattuada -- ETH Zurich \\
Chris Hawblitzel -- Microsoft Research \\
Bryan Parno, Yi Zhou, Chanhee Cho, Travis Hance -- Carnegie Mellon University \\
Jon Howell -- VMware Research}

\begin{document}

\maketitle

This \textbf{30 minute} talk will introduce Verus, an SMT-based tool we are building to
verify full functional correctness of systems code written in Rust. I am
planning to demonstrate Verus via live demos interspersed in the talk:
the examples in this proposal are samples of the snippets that I will
verify with Verus in the demos.

\section{Introduction}\label{introduction}

The Rust programming language, with its combination of high-level
features, memory safety, and low-level ``unsafe'' primitives has proven
to be an excellent tool to write high-performance systems software, with
direct interactions with the hardware.

When looking at Rust in the context of software verification
we observed that functions using the safe subset of Rust without
interior mutability are pure, deterministic transformations, even when
they manipulate mutable references (as they can be logically rewritten
as consuming the initial value of the reference, and returning the new
value). By statically preventing aliased mutable references Rust
relieves the burden on the developer of reasoning about data races; when
verifying imperative code in languages that do not restric aliasing, the
developer similarly has to explicitly manage the heap and potential
aliasing. In our experience, a Rust-like linear type system can be an
excellent aid to Floyd-Hoare SMT-based verification of executable code.

We leverage these intuitions in the design of our new tool for SMT-based
verification of Rust code, Verus.
We are building Verus to efficiently verify full functional correctness
of low-level systems code written in a safe Rust dialect that supports
expressing specifications and proofs.

Figure~\ref{fig:fibo} shows a snippet of a Rust
function~\clref{clnum:fn-fibo-impl} verified with Verus.

\section{Verus' design}\label{verus-design}

Verus leverages the fact that safe Rust has a natural, direct encoding
in SMT (for code without interior mutability, which we need to address
separately). Algebraic data types (ADTs) and sound generics complement
safe Rust to make it a solid basis for a mathematical language for
expressing specifications and proofs. In Figure~\ref{fig:fibo},
the \textcode{fibo} function~\clref{clnum:fn-fibo} is a pure mathematical definition of
the n-th Fibonacci number. When functions represent theorems
(like \textcode{lemma\_fibo\_is\_monotonic} \clref{clnum:fn-lemma-monotonic}),
expressing complex proofs like lemmas-by-induction is straightforward
via a simple recursive call~\clref{clnum:lemma-monotonic-induct}.

\begin{figure}[H]

\begin{lstlisting}[language=Rust]
#[spec]
fn fibo(n: nat) -> nat { %*\clnum\label{clnum:fn-fibo}*
  decreases(n);
  if n == 0 { 0 } else if n == 1 { 1 }
  else { fibo(n - 2) + fibo(n - 1) }
}

#[proof]
fn lemma_fibo_is_monotonic(i:nat, j:nat) { %*\clnum\label{clnum:fn-lemma-monotonic}* 
  requires(%*\clnum\label{clnum:requires-clause}* i<=j);
  ensures(%*\clnum\label{clnum:ensures-clause}* fibo(i) <= fibo(j));
  decreases(j-i);

  if i<2 && j<2 { } else if i==j { }
  else if i==j-1 {
    reveal_with_fuel(fibo, 2);
    lemma_fibo_is_monotonic(i, j-1); %*\clnum\label{clnum:lemma-monotonic-induct}*
  } else {
    lemma_fibo_is_monotonic(i, j-1);
    lemma_fibo_is_monotonic(i, j-2);
  }
}

#[spec]
fn fibo_fits_u64(n: nat) -> bool { %*\clnum\label{clnum:fn-fibo-fits}*
  fibo(n) <= u64::MAX } 

#[exec]
fn fibo_impl(n: u64) -> u64 { %*\clnum\label{clnum:fn-fibo-impl}* 
  requires(%*\clnum\label{clnum:requires-clause-impl}* fibo_fits_u64(n));
  ensures(%*\clnum\label{clnum:ensures-clause-impl}* |result: u64| result == fibo(n));
  if n == 0 { return 0; } 
  let mut prev: u64 = 0; let mut cur: u64 = 1;
  let mut i: u64 = 1;
  while i < n {
    invariant([%*\clnum\label{clnum:invariant-clause}*
      i > 0, i <= n,
      fibo_fits_u64(n as nat), fibo_fits_u64(i as nat),
      cur == fibo(i), prev == fibo(i as nat - 1),
    ]);
    let new_cur = cur + prev;
    prev = cur; cur = new_cur; i = i + 1;
    lemma_fibo_is_monotonic(i, n);
  }
  cur
}
\end{lstlisting}

\caption{A proof of correcntess of a function computing the n-th
Fibonacci number. We use circled letters, similar to \protect\circled{Z},
to mark points of interest in the code.}
\label{fig:fibo}
\end{figure}

Other tools, like Dafny, need complex encodings in SMT to support the
semantics of the language, and in our
experience this can be the cause of unpredictable prover performance.
We choose to prioritize efficient, straighforward encoding of Rust to Z3 in
Verus (without an intermediate language like Boogie).
Thanks to Rust's linear typing guarantees we can
often encode potentially mutable, complex objects, as immutable SMT
datatypes, rather than having to rely on complex rules and axioms. Even
objects with interior mutability that expose an ``immutable''
interface can often be directly encoded.

Additionally, Rust's memory management story removes a significant
challenge of other tools, like Dafny, that need to encode and manage the
heap to handle potential aliasing. Heap-reasoning is notoriously costly in SMT-based tools.
For example, in our experience building and verifying VeriBetrKV, a verified crash-safe
high-performance key-value store, we were able to reduce proof burden
(in terms of LOCs and verification time) by up to 30\% by integrating a
linear type system into a fork of Dafny.

This design principle of prioritizing efficient Z3 queries
by relying on safe Rust and the linear type system
comes with a necessary compromise:
we do not intend to support all Rust features and libraries,
instead excluding those that would
require complicated encoding.
\footnote{In the talk, I will highlight which benefits and
limitations this may bring in contrast with other Rust verification
tools like RustBelt and Prusti, and how these tools may complement each other.}
We do, however, plan to provide
verification-aided safe alternatives for patterns where ``unsafe''
would otherwise be required in Rust.

\section{Borrow-checking proofs}\label{borrow-checking-proofs}

As part of recent work, we recognized the value
of tracking ownership of ghost variables, which do not appear in the
emitted code, but aid verification. For example, in concurrent code,
linear ghost variables can facilitate tracking of ghost permissions to
access non-atomic memory, represent a thread's knowledge of a
concurrent, shared-memory protocol state, or represent limited
aliasing domains, for example as a mechanism to encapsulate and prove
non-linear data-structures that require aliased references such as
\texttt{Rc}. To enable using linearity to track these assertions,
Verus supports three modes for variables and functions,
\textbf{\textcode{\#{[}spec{]}}}, \textbf{\textcode{\#{[}proof{]}}}, and
\textbf{\textcode{\#{[}exec{]}}}, as shown in Figure~\ref{fig:fibo} and
Figure~\ref{fig:linear-ghost}.

Code and variables in \textbf{\textcode{\#{[}spec{]}} mode} are always
ghost and make up the pure mathematical language used to write specifications.
Examples of \textcode{spec} are the predicates \clref{clnum:fn-fibo},
\clref{clnum:fn-fibo-fits}, and the expressions in
\textcode{requires} \clref{clnum:requires-clause}\clref{clnum:requires-clause-impl},
\textcode{ensures} \clref{clnum:ensures-clause}\clref{clnum:ensures-clause-impl},
and \textcode{invariant} clauses \clref{clnum:invariant-clause}.
Code in this mode is not checked for linearity and borrow rules,
and is always erased before compilation.

Code in \textbf{\textcode{\#{[}proof{]}} mode} is ghost code used to
define lemmas (e.g. \clref{clnum:fn-lemma-monotonic}) and prove postconditions
(e.g. \clref{clnum:lemma-monotonic-induct}).
To support using linear ghost variables to track permissions and
other singly-owned tokens that can aid verification, \textcode{proof}
code supports linear types and is checked by the borrow checker; it
is also erased, like \textcode{spec}. Figure~\ref{fig:linear-ghost}
shows an example of the borrow-checker preventing \clref{clnum:linear-error}
an illegal use \clref{clnum:linear-illegal} of a linear ghost permission
\clref{clnum:linear-permission}
whose ownership had been transferred \clref{clnum:linear-transfer}.

Executable code (\textbf{\textcode{\#{[}exec{]}}} mode) is used for the
actual implementation (e.g. \clref{clnum:fn-fibo-impl}),
and abides by the same rules as regular Rust code.

\begin{figure}
\begin{lstlisting}[language=Rust]
#[exec]
fn increment(
  // a shared memory location
  counter: &PCell<u64>,
  // a linear ghost permission to access it
  #[proof] permission: &mut Permission<u64>) {
  requires(...); ensures(...);
  
  let cur = counter.borrow(permission); // read
  counter.put(cur + 1, permission);     // write
}

#[exec]
fn release(#[proof] permission: Permission<u64>)
{ ... }

#[exec]
fn start_thread(
  counter: &PCell<u64>,
  #[proof] mut permission: Permission<u64> %*\clnum\label{clnum:linear-permission}*) {
  requires(...);

  release(permission); %*\clnum\label{clnum:linear-transfer}*

  // error: borrow of moved value: `permission` %*\clnum\label{clnum:linear-error}*
  increment(counter, %*\clnum\label{clnum:linear-illegal}* &mut permission);

  assert(...); // that the counter was incremented
}
\end{lstlisting}

\caption{An example of a linear ``ghost'' permission used to access
a shared memory counter. This code is rejected by the borrow checker
because \textcode{permission} is a linear ``ghost'' value.}
\label{fig:linear-ghost}
\end{figure}

\section{Tailored solvers}

In our experience verifying large scale system software, we needed
the flexibility to write proofs in a variety of ways (e.g. by induction,
using ghost variables, using linearity). Verus enables
switching the solver's strategy for some proofs, for example by
selectively leveraging the SMT solver's bit-vector theory
while using the integer theory for the rest of the system.
We are also experimenting with support for dedicated solvers
(in addition to Z3) to dispatch certain verification conditions
for which they are more effective.
% is an important part of this for low-level OS
% coding, packet processing, run-time systems, crypto, etc.
\section{Well-engineered foundation}

An explicit goal of the project is to build a solid, well-engineered
foundation with excellent diagnostics, to aid onboarding of new users,
in the spirit of the Rust's compiler error reporting (I will demonstrate
this in the demos). A well-engineered foundation should also allow us
and others to use Verus as the basis to experiment with new verification
language features and techniques.

\end{document}
